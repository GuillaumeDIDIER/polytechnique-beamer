\section{Première section}
\begin{frame}
\frametitle{Première diapositive}
\framesubtitle{Un aperçu simple d'une diapositive très basique.}

Lorem ipsum dolor sit amet, consectetur adipiscing elit. Aliquam luctus, odio ut ultrices venenatis, mauris ex mollis magna, quis imperdiet felis arcu et tortor. \pause[3] Etiam eget purus varius, aliquam metus vitae, pharetra urna. Vivamus quis nisi in sapien tempor dictum. \pause[2] \textbf<3>{Ut laoreet aliquet justo, nec posuere eros placerat at. Nunc vel mauris in libero semper feugiat vel quis lacus.}

\end{frame}

\subsection{Première sous-section}
\begin{frame}
\frametitle{Seconde diapositive}

\begin{enumerate}
    \item Lorem ipsum dolor sit amet, consectetur adipiscing elit.
    \item Aliquam luctus, odio ut ultrices venenatis, mauris ex mollis magna, quis imperdiet felis arcu et tortor.
    \begin{enumerate}
        \item Etiam eget purus varius, aliquam metus vitae, pharetra urna.
    \end{enumerate}
\end{enumerate}
\begin{enumerate}[1.]
    \item Vivamus quis nisi in sapien tempor dictum.
    \begin{enumerate}[a)]
        \item Ut laoreet aliquet justo, nec posuere eros placerat at.
        \item Nunc vel mauris in libero semper feugiat vel quis lacus.
    \end{enumerate}
\end{enumerate}

\end{frame}

\subsection{Seconde sous-section}
\begin{frame}
\frametitle{Troisième diapositive}

\begin{itemize}
    \item<1-> Lorem ipsum dolor sit amet, consectetur adipiscing elit.
    \item<1-> Aliquam luctus, odio ut ultrices venenatis, mauris ex mollis magna, quis imperdiet felis arcu et tortor.
    \item<2-> Etiam eget purus varius, aliquam metus vitae, pharetra urna.
    \item<2-> Vivamus quis nisi in sapien tempor dictum.
    \begin{itemize}
        \item Ut laoreet aliquet justo, nec posuere eros placerat at.
        \item Nunc vel mauris in libero semper feugiat vel quis lacus.
    \end{itemize}
\end{itemize}
\end{frame}

\section{Seconde section}

\subsection{Première sous-section}
\begin{frame}{Quatrième diapositive}
\framesubtitle{Ceci est un sous-titre}

\url{https://openclassrooms.com/courses/creez-vos-diaporamas-en-latex-avec-beamer}

\begin{columns}[t]%t,c,b
    \begin{column}{0.4\textwidth}
        \begin{block}{Un bloc standard}
            Ceci est le bloc de base avec beamer.
        \end{block}
    \end{column}
    \begin{column}{0.4\textwidth}
        \begin{alertblock}{Un bloc alerte}<2>
            Ceci est un bloc d'alerte.
        \end{alertblock}
    \end{column}
\end{columns}

\begin{exampleblock}{Un bloc exemple}<1>
Ceci est un bloc d'exemple.
\end{exampleblock}

\end{frame}

\subsection[2nd ss]{Seconde sous-section}
\begin{frame}{Cinquième diapositive}
\framesubtitle{Ceci est un sous-titre}
\begin{verse}
    Environnement verse
\end{verse}

\begin{quotation}
    Environnement Quotation
\end{quotation}

\begin{quote}
    Environnement Quote
\end{quote}
\end{frame}

\begin{frame}{Visibilité}
\framesubtitle{Quand afficher chaque élément.}
\only<1>{Salut c'est only, je suis présent qu'au premier slide.\\}
\visible<2->{Salut c'est visible, je suis visible à partir du slide 2.\\}
\uncover<3->{Salut c'est uncover, je suis découvert à partir du slide 3.\\}
\invisible<2-4>{Salut c'est invisible, je serais invisible du slide 2 au slide 4.\\}
\alt<2>{Salut, je suis le alt qui sera au slide 2.\\}{Salut je suis le alt qui sera aux autres slides que la 2.\\}
\temporal<2-3>{Salut je suis le temporal visible du slide 1}{Et moi le temporal visible du slide 2 au slide 3}{Et moi le temporal visible après le slide 3}
\end{frame}